% !TEX program = xelatex

\documentclass{resume}
\usepackage{zh_CN-Adobefonts_external} % Simplified Chinese Support using external fonts (./fonts/zh_CN-Adobe/)
%\usepackage{zh_CN-Adobefonts_internal} % Simplified Chinese Support using system fonts

\begin{document}
\pagenumbering{gobble} % suppress displaying page number

\name{Xin Zhang}

\basicInfo{
  \email{zhangx@dsp.ac.cn} \textperiodcentered\ 
  \phone{(+86) 130-2120-2677} \textperiodcentered\ 
  \github[https://github.com/nce3xin]{https://github.com/nce3xin}}

\section{Education}
\datedsubsection{\textbf{University of Chinese Academy of Sciences}, \textit{Beijing}}{2015 -- 2020}
%\textit{Master student} in Electronics Engineering (EE), expected March 2016
\textit{Ph.D. candidate} in National Network New Media Engineering Research Center
\datedsubsection{\textbf{Communication University of China}, \textit{Beijing}}{2011 -- 2015}
%\textit{B.S.} in Electronics Engineering (EE)
GPA: 3.81 / 4 \quad Ranking:Top 2\% / 103 \quad Recommended Postgraduate

\section{Project Experience}
\datedsubsection{\textbf{Intelligent Router Service Platform} }{Jul. 2016 -- Mar. 2018}
\role{Project Description}{The Chinese Academy of Sciences' strategic pilot project, "Research on a new generation of information technology for China." The project builds a network of content-oriented services in real time by aggregating network edge devices and utilizing autonomous management and collaboration of nodes to provide reliable, real-time and efficient service response.}
%Brief introduction: xxx.
\begin{itemize}
  %\item 项目描述:中科院战略先导专项,“面向感知中国的新一代信息技术研究”,目标是利用网络边缘设备,提供现场、弹性、自治服务。
  \item Innovatively proposed a decentralized content dissemination system based on the network edge devices;
  \item A PageRank-based node selection algorithm is proposed, and the service rejection rate is reduced by \textbf{[4.4\%-8.2\%]};
  \item A lightweight decentralized approximation method is proposed to solve the problem of high complexity of global PageRank iterative calculation.
\end{itemize}

\datedsubsection{\textbf{Detecting Spam Reviewers for Movies on the Web}}{Jan. 2018 -- Aug. 2018}
\role{Project Description}{Detecting spam reviewers for movies based on the deep learning approach. It is required to improve the classification performance metrics such as accuracy and recall rate in the case of unbalanced samples and missing labels.}
%Brief introduction: xxx
\begin{itemize}
  \item Innovatively proposed the Temporal-Spatial Mapping method, which maps vectors to two-dimensional grayscale images and introduces CNN to extract compression features;
  \item Synthesizing new samples based on the oversampling SMOTE algorithm to reduce over-fitting risk and solve sample imbalance problem;
  \item A manual labeling method assisted by clustering is proposed to speed up the manual labeling process of new datasets;
  \item Experiments show that at least \textbf{29\%} misjudgments can be corrected, especially for spammers who deliberately imitate ordinary users.
\end{itemize}


\datedsubsection{\textbf{Packet Analysis Software Development}}{Feb. 2016 -- Sep. 2016}
\role{Project Description}{The following eight protocols are parsed from the .pcap file: ARP, IP, ICMP, RIP, OSPF, UDP, TCP, DHCP.}
%Brief introduction: xxx
\begin{itemize}
  \item Building the parsing system independently (C++);
  \item Building the parsing engine for Wireshark's capture format: .pcap file;
  \item Developing a GUI based on Qt interface.
\end{itemize}

%\datedsubsection{\textbf{Research on Edge Intelligence Technology}}{Oct. 2018 -- Mar. 2019}
%\role{Project Description}{The research on edge intelligence algorithm is mainly applied to the model learning ability problem when the terminal resources are limited. The goal is to achieve lightweight learning of deep learning algorithms, multi-terminal collaborative computing, fast convergence, low resource consumption, and rapid response capability. It is the main demand scenario for 5G and the Internet of Things.}\
%\begin{itemize}
%  \item Based on the centralized deep learning algorithms, the possibility of distributed, lightweight and deployment at the edge of the network is studied.
%  \item This topic is the research direction of my doctoral thesis, and the research is still in progress.
%\end{itemize}

\datedsubsection{\textbf{Ultra-Low Latency and Ultra-Reliable Virtual Reality}}{Oct. 2018 -- May. 2019}
\role{Project Description}{VR/AR is expected to be one of the killer applications for 5G networks. The next step in future interconnected VR/AR comes from the flexible use of computing, caching and communication resources, the so-called $C^3$ paradigm. The project presents the user's HD frames at the edge of the network through a joint active computing and caching scheme.}
\begin{itemize}
  \item Computation offload. Computing tasks related to user tracking information (such as game actions or video stream preferences) are offloaded to an edge server with high computing power and the computed video frames are returned in the downlink direction;
  \item Based on the user's future posture estimation, the video frame is proactively calculated in the remote cloud server and cached in the edge of the network or the user's HMD headset, releasing more edge servers for real-time tasks;
  \item High-definition frames are proactively calculated and cached corresponding to the user's upcoming movements, head rotations and estimated large numbers of actions;
  \item The application-specific actions and corresponding decisions are proactively predicted, as well as the upcoming actions based on the popularity of different behaviors and their impact on the VR/AR environment.
\end{itemize}



% Reference Test
%\datedsubsection{\textbf{Paper Title\cite{zaharia2012resilient}}}{May. 2015}
%An xxx optimized for xxx\cite{verma2015large}
%\begin{itemize}
%  \item main contribution
%\end{itemize}

\section{Academic Achievements}
\begin{itemize}[parsep=0.5ex]
  \item \textbf{Zhang, X.}, You, J., Xue, H., \& Wang, J. (2019). A Decentralized PageRank Based Content Dissemination Model in the Edge of Network. \textit{International Journal of Web Services Research, IJWSR} (SCI Journal, the first author)
  \item You, J., \textbf{Zhang, X.}, Lian, W., Detecting Spam Reviewers for Movies on the Web. \textit{Applied Sciences} (SCI Journal, under review, the first author)
  \item You, J., Xue, H., Zhuo, Y., \textbf{Zhang, X.}, \& Wang, J. (2017). Forecasting Service Performance on the Basis of Temporal Information by the Conditional Restricted Boltzmann Machine. \textit{IEICE Transactions on Communications.} (SCI Journal)
  \item Patent:"A Decentralized PageRank Acceleration Method Based on Similarity Estimation" (First author),Application number:BDI170716
  \item Patent:"A Content distribution method based on the dynamic adjustment of coverage rate in node self-organizing network",Application number:201810027211.2
\end{itemize}

\section{Honors and Awards}
\datedline{Silver Medal, $13^{th}$ place, Huawei Software Elite Challenge}{2019}
\datedline{Excellent Student Cadre,University of Chinese Academy of Sciences}{2017}
\datedline{Merit Studen, University of Chinese Academy of Sciences}{2016}
\datedline{Outstanding Party Member, University of Chinese Academy of Sciences}{2016}
\datedline{Outstanding Graduates of Beijing, Communication University of China}{2015}
\datedline{Scholarship of CCTV, Central People's Broadcasting Station, China Radio International,Communication University of China}{2012-2013}
\datedline{Japanese TBS TV Scholarship, Communication University of China}{2014}



\section{Skills}
\begin{itemize}[parsep=0.5ex]
  \item CET-6, fluent English reading, writing and communication skills;
  \item Familiar with C++, Python and linux systems;
  \item Familiar with basic data structures and algorithms, with good programming style;
  \item Proficiency in basic deep learning algorithms and deep learning tools such as PyTorch.
\end{itemize}

\section{Social experience}
\begin{itemize}[parsep=0.5ex]
  \item \datedline{Minister of the Graduate School of the Institute of Acoustics, Chinese Academy of Sciences}{Sep. 2016 -- Aug. 2017}
  \item \datedline{Volunteer of the Public Science Open Day of the Chinese Academy of Sciences}{$20^{th}$ May. 2017}
\end{itemize}

\section{Personal evaluation}
\begin{itemize}[parsep=0.5ex]
  \item Strong sense of responsibility, teamwork awareness, positive work attitude;
  \item Strong learning ability and willing to accept new things;
  \item Certain artistic skills:
  \begin{itemize}[parsep=0.5ex]
    \item[*] Grade 8 of Piano, Grade 9 of Electronic Piano;
    \item[*] Skilled in hand drawing and graphic design, proficient in Adobe Photoshop, Adobe Illustrator and Adobe After Effects. See my \href{https://nce3xin.github.io/design-portfolio/}{\textbf{portfolio}};
    \item[*] Video editing, once edited technical videos, helped the lab's 5G standard to be successfully established at the ITU International Conference;
    \item[*] Love table tennis, billiards, music, photography. Once participated in the table tennis team competition and won the first place.
  \end{itemize}
  \item Personal website: \href{https://code.nce3xin.me/}{https://code.nce3xin.me/}
\end{itemize}

%% Reference
%\newpage
%\bibliographystyle{IEEETran}
%\bibliography{mycite}
\end{document}
