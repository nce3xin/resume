% !TEX program = xelatex

\documentclass{resume}
\usepackage{zh_CN-Adobefonts_external} % Simplified Chinese Support using external fonts (./fonts/zh_CN-Adobe/)
%\usepackage{zh_CN-Adobefonts_internal} % Simplified Chinese Support using system fonts

\begin{document}
\pagenumbering{gobble} % suppress displaying page number

\name{张欣}

\basicInfo{
  \email{zhangx@dsp.ac.cn} \textperiodcentered\ 
  \phone{(+86) 130-2120-2677} \textperiodcentered\ 
  \github[https://github.com/nce3xin]{https://github.com/nce3xin}}

\section{教育背景}
\datedsubsection{\textbf{中国科学院大学} \quad 国家网络新媒体工程技术研究中心 \quad 硕博连读}{2015 -- 2020}
%\textit{Master student} in Electronics Engineering (EE), expected March 2016
博士研究生 \quad 专业:信号与信息处理 \quad 研究方向:智能边缘计算
\datedsubsection{\textbf{中国传媒大学}\quad 信息工程学院\quad 工学学士}{2011 -- 2015}
%\textit{B.S.} in Electronics Engineering (EE)
GPA: 3.81 / 4 \quad 专业排名:Top 2\% / 103 \quad 保研

\section{项目经历}
\datedsubsection{\textbf{智能路由器项目} }{2016年7月 -- 2018年3月}
\role{项目描述}{中科院战略先导专项,“面向感知中国的新一代信息技术研究”。该项目面向互联网视频分发服务领域,针对目前视频运营商高额骨干带宽流量压力及用户服务质量保障的需求,基于边缘计算服务技术,利用大规模智能路由器设备的统一管理与调度,实现用户近场内容分发服务、有效减少骨干带宽流量的平台产品。}
%Brief introduction: xxx.
\begin{itemize}
  %\item 项目描述:中科院战略先导专项,“面向感知中国的新一代信息技术研究”,目标是利用网络边缘设备,提供现场、弹性、自治服务。
  \item 针对边缘网络规模极大,中心服务器指导内容扩散和资源调度时负载过大的问题,基于边缘设备的局部自治,模拟谣言传播的方式,提出了一种去中心化的内容扩散系统;
  \item 内容扩散过程中选择哪个节点缓存内容的副本对扩散后的系统性能会有较大影响。为此提出一种基于加权分布式PageRank算法的启发式缓存节点选择策略。实验证明,此策略可使系统的服务拒绝率降低\textbf{[4.4\%-8.2\%]};
  \item 网络节点的一大特点是对庞大的全局拓扑信息不可得,因此导致无法计算全局PageRank的问题。为了解决此问题,提出了一种基于轻量级分布式PageRank近似估计算法的解决方案。
\end{itemize}

\datedsubsection{\textbf{面向超低时延和超高可靠的虚拟现实}}{2018年10月 -- 2019年5月}
\role{项目描述}{VR/AR预计将成为5G网络的杀手级应用之一。该项目提出一种优化框架“联合主动计算和缓存方案”,在可靠性和时延限制的情况下,最大化视频帧的成功交付。基于用户的运动姿态预测,主动计算未来一段时间窗内对应的视频帧,并将其缓存在网络边缘设备或用户的头戴设备中,释放更多的边缘服务器来处理受随机到达的游戏动作影响的视频帧。对比全部实时计算的基线,时延降低一倍。}
\begin{itemize}
  \item 计算卸载:将与用户跟踪信息(如游戏动作或视频流偏好)相关的计算任务卸载到具有高计算能力的边缘服务器,并在下行链路方向上返回计算后的视频帧;
  \item 主动计算和缓存与用户即将进行的移动和头部旋转以及估计的大量动作相对应的高清帧;
  \item 计算任务的调度遵循不同的优先级。优先级最高的是实时计算,目标是处理受随机到达的游戏动作影响的视频帧;之后,根据计算和存储资源的限制,计算和缓存未来的视频帧。
\end{itemize}

\datedsubsection{\textbf{网络数据包解析引擎开发}}{2016年2月 -- 2016年9月}
%\role{项目描述}{从.pcap文件中解析以下8种协议内容:ARP, IP, ICMP, RIP, OSPF, UDP, TCP, DHCP}
\role{项目描述}{根据网络协议分析流程对数据包在TCP/IP各层协议中进行实际解包分析,为网络协议分析和还原提供技术手段。}
%Brief introduction: xxx
\begin{itemize}
  \item 独立搭建网络数据包解析系统(C++语言);
  \item 独立开发.pcap文件格式解析系统;
  \item 基于Qt接口独立开发图形界面;
  \item 基于报文解析树的形式展示最终解析结果。
\end{itemize}

\datedsubsection{\textbf{电影评论水军检测系统}}{2018年1月 -- 2018年8月}
\role{项目描述}{基于神经网络算法对网络上的水军进行检测,目标是提升整体以及各类别的准确率、召回率等分类性能指标。}
%Brief introduction: xxx
\begin{itemize}
  \item 创新性提出了时间空间映射(Temporal-Spatial Mapping)方法,将用户在时间轴上的一维转发数据通过灰度图的方式映射到二维空间,再用CNN从稀疏的矩阵中提取压缩后的特征;
  \item 基于过采样的SMOTE算法合成新样本,降低过拟合风险,解决类别样本不均衡问题;
  \item 提出基于聚类辅助的手动标注方法,加快新数据集的手动标注过程,解决标签缺失问题;
  \item 实验表明,至少\textbf{29\%}的误判可以得到纠正,特别是对于故意模仿普通用户的垃圾评论者。
\end{itemize}




%\datedsubsection{\textbf{边缘智能技术研究}}{2018年10月 -- 2019年3月}
%\role{项目描述}{关于边缘智能算法的研究主要应用于终端设备资源受限情况下的模型学习能力问题。目标是实现深度学习算法的轻量化,多终端协作计算,收敛速度快,消耗资源少,提供快速响应能力,是5G的主要需求场景。}
%\begin{itemize}
%  \item 以中心化深度学习算法为基础,对其分布式、轻量化、网络边缘部署的可能性进行研究;
%  \item 该课题为本人博士论文研究方向,研究仍在进行中。
%\end{itemize}



% Reference Test
%\datedsubsection{\textbf{Paper Title\cite{zaharia2012resilient}}}{May. 2015}
%An xxx optimized for xxx\cite{verma2015large}
%\begin{itemize}
%  \item main contribution
%\end{itemize}

\section{科研成果}
\begin{itemize}[parsep=0.5ex]
  \item \textbf{Zhang, X.}, You, J., Xue, H., \& Wang, J. (2019). A Decentralized PageRank Based Content Dissemination Model in the Edge of Network. \textit{International Journal of Web Services Research, IJWSR} (SCI期刊,第一作者)
  \item You, J., \textbf{Zhang, X.}, Lian, W., Detecting Spam Reviewers for Movies on the Web. \textit{Applied Sciences} (SCI期刊在投,学生一作)
  \item You, J., Xue, H., Zhuo, Y., \textbf{Zhang, X.}, \& Wang, J. (2017). Forecasting Service Performance on the Basis of Temporal Information by the Conditional Restricted Boltzmann Machine. \textit{IEICE Transactions on Communications.} (SCI期刊)
  \item 专利:“一种基于相似度估计的分布式PageRank加速方法”(第一作者),申请号:BDI170716
  \item 专利:“一种节点自组网中基于覆盖率动态调整的内容分发方法”,申请号:201810027211.2
\end{itemize}

\section{荣誉奖励}
\datedline{华为软件精英挑战赛二等奖,$13^{th}$ place}{2019}
\datedline{优秀学生干部,中国科学院大学}{2017}
\datedline{三好学生,中国科学院大学}{2016}
\datedline{优秀党员,中国科学院大学}{2016}
\datedline{北京市优秀毕业生,中国传媒大学优秀毕业生}{2015}
\datedline{中央三台奖学金,中国传媒大学}{2012-2013}
\datedline{日本TBS电视台奖学金,中国传媒大学}{2014}



\section{专业技能}
\begin{itemize}[parsep=0.5ex]
  \item 通过CET-6,能够流畅进行英文阅读、写作和交流;
  \item 熟练掌握C++, Python等编程语言以及Linux系统;
  \item 熟练掌握基本数据结构和算法,有良好的编程风格;
  \item 熟练掌握深度学习领域基本算法以及PyTorch等深度学习工具。
\end{itemize}

\section{实践经历}
\begin{itemize}[parsep=0.5ex]
  \item \datedline{中科院声学研究所研究生会部长}{2016年9月-2017年8月}
  \item \datedline{中国科学院“公众科学开放日”志愿者}{2017年5月20日}
\end{itemize}

\section{个人评价}
\begin{itemize}[parsep=0.5ex]
  \item 责任心强,有团队合作意识,工作态度积极;
  \item 学习能力强,乐于接受新鲜事物;
  \item 具有一定艺术技能:
  \begin{itemize}[parsep=0.5ex]
    \item[*] 钢琴八级、电子琴九级;
    \item[*] 擅长手绘与平面设计,熟练运用Adobe Photoshop、Adobe Illustrator与Adobe After Effects;
    \item[*] 视频剪辑,曾剪辑技术视频协助实验室提出的5G标准在ITU国际会议上顺利立项;
    \item[*] 热爱乒乓球、台球、音乐、摄影。乒乓球曾获团体比赛第一名。
  \end{itemize}
  \item 个人网站:\href{https://code.nce3xin.me/}{https://code.nce3xin.me/}
\end{itemize}

%% Reference
%\newpage
%\bibliographystyle{IEEETran}
%\bibliography{mycite}
\end{document}
